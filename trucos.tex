% Este fichero es parte del N�mero 3 de la Revista Occam's Razor
% Revista Occam's Razor N�mero 3
%
% (c)  2007, 2008, The Occam's Razor Team
%
% Esta obra est� bajo una licencia Reconocimiento 2.5 Espa�a de Creative
% Commons. Para ver una copia de esta licencia, visite 
% http://creativecommons.org/licenses/by/2.5/es/
% o envie una carta a Creative Commons, 171 Second Street, Suite 300, 
% San Francisco, California 94105, USA.
%
% Secci�n Trucos


\rput(3,-2.5){\resizebox{!}{6cm}{{\epsfbox{images/general/varita3.eps}}}}
\begin{flushright}
\msection{red}{black}{0.1}{TRUCOS}

\mtitle{6cm}{Con un par... de l�neas}

\msubtitle{8cm}{Chuletillas para hacer cosas m� r�pido}

{\sf por Tamariz el de la Perd�z}

{\psset{linecolor=black,linestyle=dotted}\psline(-10,0)}

\end{flushright}

\vspace{4mm}

\begin{multicols}{2}
\raggedcolumns


\sectiontext{white}{black}{L�NEA DE COMANDOS AVANZADA PARA CUALQUIER
PROGRAMA DE CONSOLA}
\hrule
\vspace{2mm}

Utilizando la herramienta {\tt cle} podemos a�adir funcionalidades
avanzadas de edici�n a cualquier programa en modo consola.

Probad, por ejemplo (previa instalaci�n del paquete claro):

\vspace{2mm}

\hrule
{\footnotesize
\begin{verbatim}
occam@razor # cle nc localhost 110
\end{verbatim}
}
\hrule

\vspace{2mm}

Ahora, tras escribir unos comandos, pulsad el cursor arriba, o la
tecla inicio para editar la l�nea actual que est�is editando... Impresionante!.


\sectiontext{white}{black}{CIFRANDO TU NUEVO DISCO}
\hrule
\vspace{2mm}

Los ``device mappers'' proporcionados por los kernel Linux 2.6
permiten hacer cosas bastante interesantes. Una de ellas es el cifrado
de dispositivos de una forma sencilla. Si has instalado una
distribuci�n recientemente, esto funcionar� directamente, sino,
tendr�s que instalar algunos paquetes para poder utilizar los ``device mappers''.

\vspace{2mm}
\hrule
{\footnotesize
\begin{verbatim}
occam@razor # cfisk /dev/sdb
occam@razor # crypsetup create -y crypt /dev/sdb1 
Password:
Password:
occam@razor # mfs.ext3 /dev/mapper/crypt
occam@razor # mount /dev/mapper/crypt \
> /mnt/mi_disco_seguro
occam@razor #
\end{verbatim}		   
}
\hrule
\vspace{2mm}

La secuencia de comandos anterior nos permite formatear el disco duro
(cfdisk), activar el device mapper para una de las particiones
(cryptsetuo) formatearla (mkfs.ext3) y montar la nueva partici�n cifrada.

Eso solo lo ten�is que hacer la primera vez. Para su uso normal solo ten�is que hacer lo siguiente:

\vspace{2mm}
\hrule
{\footnotesize
\begin{verbatim}
occam@razor # crypsetup create crypt /dev/sdb1 
occam@razor # mount /dev/mapper/crypt \
> /mnt/mi_disco_seguro
occam@razor #
\end{verbatim}		   
}
\hrule
\vspace{2mm}



\sectiontext{white}{black}{VOLCADO HEXADECIMAL}
\hrule
\vspace{2mm}

La herramienta {\tt od} es una de esas grandes desconocidas en los
sistemas UNIX, que merece la pena conocer dada su
versatilidad. B�sicamente, esta herramienta nos permite realizar
volcados de datos binarios.

Probablemente los par�metros que utilic�is m�s a menudo ser�n estos:

\vspace{2mm}

\hrule
{\scriptsize
\begin{verbatim}
occams@razor # od -Ax -tx1z /bin/ls
\end{verbatim}
}
\hrule
\vspace{2mm}

El primer par�metro -Ax indica que la base del volcado ser�
hexadecimal (otros valores posibles son d para decimal y o para
octal). A continuaci�n indicamos el formato deseado para el volcado,
en este caso {\tt x1z} es la cadena elegida. En ella, {\tt x1} indica
a od que vuelque cada l�nea de datos como valores hexadecimales de 1
byte. Si os interesa, por lo que sea, buscar valores de 16 � 32 bits
en vuestro chorro binario cambiad 1 por 2 � 4. La {\tt z} final indica
a od que, adem�s, vuelque esos datos como una cadena de caracteres.

\vspace{2mm}

\hrule
{\scriptsize
\begin{verbatim}
occams@razor # nc servicio 10000 | od -Ax -tx1z /bin/ls
\end{verbatim}
}
\hrule
\vspace{2mm}

Cuando no hay otra cosa a mano, es una opci�n para depurar protocolos
binarios :).

\vspace{2mm}

\sectiontext{white}{black}{CIFRADO DE FICHEROS}
\hrule
\vspace{2mm}

Cifrar ficheros es realmente sencillo usando gpg. Alguna gente cree
que para usar gpg tiene que generar distintos tipos de claves que no
sabe muy bien como manejar... Bien, eso es cierto, para utilizar toda
la potencia de gpg, pero si solo quieres cifrar un fichero, gpg ofrece
un modo de cifrado sim�trico... el de la palabra secreta de toda la
vida.

Lo �nico que ten�is que hacer para cifrar y posteriormente descifrar
un fichero es: 

\vspace{2mm}

\hrule
{\scriptsize
\begin{verbatim}
occams@razor $ gpg -c mi_fichero.tgz
occams@razor $ gpg  mi_fichero.tgz.gpg
\end{verbatim}
}
\hrule 
\vspace{2mm}

Por defecto, gpg usa el algoritmo CAST5, pero pod�is utilizar
cualquiera disponible en vuestra versi�n. Para saber que algoritmos
pod�is usar:


\vspace{2mm}

\hrule
{\scriptsize
\begin{verbatim}
occams@razor $ gpg --version
(...)
Supported algorithms:
Pubkey: RSA, RSA-E, RSA-S, ELG-E, DSA
Cypher: 3DES, CAST5, BLOWFISH, AES, AES192, AES256, TWOFISH
Hash: MD5, SHA1, RIPEMD160, SHA256, SHA384, SHA512
Compression: Uncompressed, ZIP, ZLIB, BZIP2
occams@razor $
\end{verbatim}
}
\hrule 
\vspace{2mm}

Como pod�is ver, DES ya ni aparece :)

%% Call for tricks

\vspace{2mm}

\sectiontext{white}{black}{UNDO EN VIM}
\hrule
\vspace{2mm}

S�, se nos olvid� comentarlo en el art�culo y ya no ten�amos
sitio. Pulsad ``u'' en modo comando para deshacer y ``CTRL+R'' para
rehacer. ;)

\vspace{2mm}

\colorbox{introcolor}{
\begin{minipage}{.9\linewidth}{
\textbf{\textsf{Env�a tus trucos}}

\vspace{1mm}

\textsf{Puedes enviarnos esos trucos que usas a diario para compartirlos con el resto de lectores a la direcci�n: }

\vspace{2mm}

\texttt{occams-razor@uvigo.es}
}
\end{minipage}
}

\raggedcolumns
\pagebreak

\vspace{6cm}
\end{multicols}

\clearpage
\pagebreak

